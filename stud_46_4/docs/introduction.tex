% !TeX spellcheck = ru_RU
\newpage
\chapter*{Введение}
\addcontentsline{toc}{chapter}{Введение}
По мере развития вычислительных систем программисты столкнулись с необходимостью производить параллельную обработку данных для улучшения отзывчивости системы, ускорения производимых вычислений и рационального использования вычислитлельных мощностей. Благодаря развитию процессоров стало возможным использовать один процессор для выполнения нескольких параллельных операций, что дало начало термину <<многопоточность>>.

\textbf{Целью данной лабораторной работы} является изучение принципов и получение навыков организации  параллельного выполнения операций на примере сервера раздачи статической информации.\\

\textbf{Задачи данной лабораторной работы:}
\begin{enumerate}
	\item изучение основы распараллеливания вычислений;
	\item проведение сравнительного анализа по времени работы алгоритма выделения наиболее информативных терминов набора документов на основе документной частоты с использованием многопоточности и без нее;
	\item проведение анализа зависимости времени выполнения выше озвученного алгоритма от количества потоков, участвующих в обработке;
	\item описание и обоснование полученных результатов в отчете о выполненной лабораторной работе.
\end{enumerate}
