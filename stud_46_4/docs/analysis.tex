% !TeX spellcheck = ru_RU
\chapter{Аналитическая часть}

В данном разделе была представлена информация о многопоточности и исследуемом алгоритме выделения терминов для классификации.

\section{Многопоточность}
\textbf{Многопоточность} \cite{threads} --- это способность центрального процессора (ЦП) обеспечивать одновременное выполнение нескольких потоков в рамках использования ресурсов одного процессора. Поток -- последовательность инструкций, которые могут исполняться параллельно с другими потоками одного и того же породившего их процесса.

Процессом называют программу в ходе своего выполнения. Каждый процесс может состоять из одного или нескольких потоков.
Процесс завершается тогда, когда все его потоки заканчивают работу.
Каждый поток в операционной системе является задачей, которую должен выполнить процессор. Сейчас большинство процессоров умеют выполнять несколько задач на одном ядре, создавая дополнительные, виртуальные ядра, или же имеют несколько физических ядер. Такие процессоры называются многоядерными. 

При использовании потоков возникает проблема множественного доступа к информации. Необходимо обеспечить невозможность записи в одну ячейку памяти из нескольких потоков одновременною

Таким образом, необходимо использовать примитивы синхронизации обращения к данным. Одним из таких примитивов является мьютекс. 
Мьютекс может быть захвачен для работы в монопольном режиме или освобождён. 
Так, если 2 потока одновременно пытаются захватить мьютекс, захват произведет только один, а другой будет ждать освобождения.. 

\textit{Критическая секция} --- набор инструкций, выполняемый между захватом и высвобождением мьютекса. 
Поскольку в то время, пока мьютекс захвачен, остальные потоки, требующие выполнения критической секции, ждут освобождения мьютекса, требуется разрабатывать программное обеспечение таким образом, чтобы критическая секция была минимизирована по времени выполнения для сокращения задержек.

\section{Документная частота}
Самая простая и вполне эффективная \cite{DF} техника оценки «важности терминов для классификации» основана на наблюдении того, что значительное число терминов коллекции встречаются в малом числе документов, а наибольшую информативность имеют термины со средней или даже высокой частотой, если предварительно были удалены стоп-слова. Данная техника может применяться как единственная, так и предшествовать другой технике отбора признаков.

Таким образом, алгоритм выделения наиболее информативных терминов (из выборки документов) состоит на подсчете документной частоты (DF) --- количества документов, в которых встречается термин $t_k$ и выбором терминов с наибольшими значениями DF.

В данной лабораторной работе проводится распараллеливание алгоритма выделения терминов из выборки текстов на основе документной частоты. Для этого все документы поровну распределяются между всеми потоками. 

В качестве одного из аргументов каждый поток получает выделенный для него массив счетчиков DF длины L, где L --- это мощность алфавита. Так как каждый массив передается в монопольное использование каждому потоку, не возникает конфликтов доступа к разделяемым ячейкам памяти, следовательно, в использовании средства синхронизации в виде мьютекса нет нужды.


\section*{Вывод}
В данном разделе была представлена информация о многопоточности и исследуемом алгоритме.