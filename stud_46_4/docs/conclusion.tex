% !TeX spellcheck = ru_RU
\chapter*{Заключение}
\addcontentsline{toc}{chapter}{Заключение}

В ходе выполнения лабораторной работы было выявлено, что в результате использования многопоточной реализации время выполнения процесса может как улучшиться, так и ухудшиться в зависимости от количества используемых потоков.

Выборка из результатов замеров времени (для 8192 документов):
\begin{itemize}
	\item однопоточный процесс --- 2066 мкс;
	\item один дополнительный поток, выполняющий все вычисления --- 2450 мкс;
	\item 8 потоков (лучший результат) --- 956 мкс, что в 2,16 раз быстрее выполнения однопоточного процесса;
	\item 16 потоков --- 1260 мкс, что в 1,64 раз быстрее выполнения однопоточного процесса;
	\item 32 потока  --- 2014 мкс, что сравнимо с временем выполнения однопоточного процесса;
	\item 64 потока (худший результат) --- 3711 мкс, что в 1,80 раз медленнее выполнения однопоточного процесс.
\end{itemize}

При слишком большом значении потоков (более 8 для устройства, на котором проводилось тестирование) затраты на содержание потоков превышают преимущество от использования многопоточности и время выполнения по сравнению с лучшим результатом (для 8 потоков) растут.

Пусть $x$ --- количество потоков, $MAX$ --- максимальное количество потоков, разрешенное для одного процесса.
Влияние количества потоков на время выполнения по сравнению с однопоточным процессом:
\begin{itemize}
	\item $x \in [2; 32)$ --- улучшение времени выполнения;
	\item $x \in {32}$  --- влияние на время выполнения незначительно;
	\item $x \in (32, MAX)$  --- ухудшение времени выполнения.
\end{itemize}

Цель, поставленная в начале работы, была достигнута. Кроме того были достигнуты все поставленные задачи:
\begin{itemize}
	\item были изучены основы распараллеливания вычислений;
	\item был проведен сравнительный анализ по времени работы алгоритма выделения наиболее информативных терминов набора документов на основе документной частоты с использованием многопоточности и без нее;
	\item был проведен сравнительный анализ зависимости времени выполнения выше озвученного алгоритма от количества потоков, участвующих в обработке;
	\item был подготовлен отчет о лабораторной работе.
\end{itemize}
