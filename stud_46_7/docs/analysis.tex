% !TeX spellcheck = ru_RU
\chapter{Аналитическая часть}

В этом разделе будет представлена информация о задаче коммивояжера, способах ее решения --- методе полного перебора и методе на основе муравьиного алгоритма, а также выдвинуты требования к разрабатываемому программному продукту.

%\section{Задача коммивояжера}
Задача коммивояжера одна из самых важных задач всей транспортной логистики, в которой рассматриваются вершины графа, а также матрица смежности (для расстояния между вершинами)\,\cite{task}. 
Задача заключается в том, чтобы найти такой порядок посещения вершин графа, при котором путь будет минимален, каждая вершина будет посещена лишь один раз, а возврат произойдет в начальную вершину. Согласно варианту задания рассматривается разновидность задачи с незамкнутым маршрутом, т.е. возврат в начальную вершину в конце маршрута производиться не должен.

%\section{Метод полного перебора для решения задачи коммивояжёра}

Полный перебор для задачи коммивояжера\,\cite{full-comb} имеет высокую сложность алгоритма ($n!$), где $n$ --- количество городов. 
Суть метода заключается в полном переборе всех возможных путей в графе и выбор наименьшего из них. 
Решение будет получено, но имеются большие затраты по времени выполнения на уже небольшом количестве вершин в графе.

\section{Метод на основе муравьиного алгоритма}

Муравьиный алгоритм\,\cite{full-comb} --- метод решения задачи оптимизации, основанный на принципе поведения колонии муравьев.

Муравьи действуют, руководствуясь органами чувств. 
Муравьи используют непрямой обмен информацией через окружающую среду, посредством феромона.
Каждый муравей оставляет на своем пути феромоны, чтобы другие могли ориентироваться. 
Наибольшее количество феромона остается на наиболее посещаемом пути, посещаемость же может быть связана с длинами ребер.

У муравья есть несколько чувств:
\begin{enumerate}[label=\arabic*)]
	\item обоняние --- способность чуять феромон.
	\item память --- способность запомнить пройденный маршрут;
	\item зрение --- способность определить привлекательность ребра ($\eta = \frac{1}{D}$, где D --- длина ребра);
\end{enumerate}

Вероятность перехода муравья по ребру $ij$ вычисляется по формуле
\begin{equation}
	\label{posib}
	p_{k,ij} = \begin{cases}
		\frac{\eta_{ij}^{\alpha}\cdot\tau_{ij}^{\beta}}{\sum_{q\notin J_k} \eta^\alpha_{iq}\cdot\tau^\beta_{iq}}, j \notin J_k \\
		0, j \in J_k
	\end{cases}
\end{equation}
где $a$ --- параметр влияния длины пути, $b$ --- параметр влияния феромона, $\tau_{ij}$ --- количество феромонов на ребре $ij$, $\eta_{ij}$ --- привлекательность ребра $ij$, $J_k$ --- список посещенных за текущий день городов.

После завершения движения всех муравьев (ночью, перед наступлением следующего дня), феромон обновляется по формуле
\begin{equation}
	\tau_{ij}(t+1) = \tau_{ij}(t)\cdot(1-p) + \Delta \tau_{ij}(t).
\end{equation}
При этом
\begin{equation}
	\Delta \tau_{ij}(t) = \sum_{k=1}^N \Delta \tau^k_{ij}(t),
\end{equation}
где
\begin{equation}
	\label{ant_pheromone}
	\Delta\tau^k_{ij}(t) = \begin{cases}
		Q/L_{k}, \textrm{ребро посещено муравьем $k$ в текущий день $t$,} \\
		0, \textrm{иначе}
	\end{cases}
\end{equation}

Поскольку вероятность перехода в заданную точку \ref{posib} не должна быть равна нулю, необходимо обеспечить неравенство $\tau_{ij} (t)$ нулю посредством введения минимального значения феромона $\tau_{min}$ и в случае, если $\tau_{ij} (t+1)$ принимает значение, меньшее $\tau_{min}$, откатывать феромон до этой величины. 

Путь выбирается по следующей схеме.
\begin{enumerate}[label=\arabic*)]
	\item Каждый муравей имеет список запретов --- список уже посещенных городов (вершин графа).
	\item Муравьиное зрение отвечает за эвристическое желание посетить вершину.
	\item Муравьиное обоняние отвечает за ощущение феромона на определенном пути (ребре). При этом количество феромона на пути (ребре) в день $t$ обозначается как $\tau_{i, j} (t)$.
	\item После прохождения определенного ребра муравей откладывает на нем некоторое количество феромона, которое показывает оптимальность сделанного выбора, это количество вычисляется по формуле \eqref{ant_pheromone}.
\end{enumerate}

\section*{Вывод}
В данном разделе была рассмотрена задача коммивояжёра, а также способы её решения --- полный перебор и муравьиный алгоритм. 