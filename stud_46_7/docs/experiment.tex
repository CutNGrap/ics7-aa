% !TeX spellcheck = ru_RU
\chapter{Исследовательская часть}
В данном разделе будут приведены примеры работы программы, а также проведен сравнительный анализ алгоритмов при различных ситуациях на основе полученных данных.


\section{Технические характеристики}

Ниже приведены технические характеристики устройства, на котором было проведено измерение времени работы ПО:

\begin{itemize}
	\item операционная система Windows 10 Домашняя Версия 21H1 \cite{windows} x86\_64;
	\item оперативная память 8 Гбайт 2133 МГц;
	\item процессор Intel Core i5-8300H с тактовой частотой 2.30 ГГц \cite{intel}, 4 физических ядра, 8 логических ядер.
\end{itemize}

\section{Демонстрация работы программы}

На рисунке \ref{img:example} представлен пример результата работы программы. Пользователь, указывая соответствующий пункт меню и название файла с матрицой смежностей, запускает поиск минимального расстояния с помощью алгоритма полного перебора. 
На экран выводится результат работы алгоритма.

\img{0.6}{example}{Пример работы программы}
\pagebreak

\section{Результаты замеров времени}

Время работы было замерено с помощью функции \textit{high\_resolution\_clock(...)} из заголовочного файла \textit{chrono} на \textit{С++}.

Замеры проводились для разного размера матриц, чтобы определить, когда наиболее эффективно использовать муравьиный алгоритм.

Результаты замеров приведены в таблице \ref{table_measuring}, время приведено в миллисекундах.

\newcolumntype{d}[1]{D{.}{.}{-1}}

\begin{table}[h!]
	\begin{center}
		\begin{threeparttable}
			\captionsetup{justification=raggedright,singlelinecheck=off}
			\caption{Результаты замеров времени (в мс)}
			\label{table_measuring}
			\begin{tabular}{|c|c|c|}
				\hline
				Размер матрицы & Полный перебор & Муравьиный алгоритм \\
				\hline
				2 & 0.012 & 3.984\\
				\hline
				3 & 0.027 & 7.487\\
				\hline
				4 & 0.103 & 13.875\\
				\hline
				5 & 0.494 & 24.813\\
				\hline
				6 & 3.252 & 41.180\\
				\hline
				7 & 20.901 & 66.088\\
				\hline
				8 & 170.166 & 96.923\\
				\hline
				9 & 1623.428 & 134.993\\
				\hline
			\end{tabular}
		\end{threeparttable}
	\end{center}
\end{table}

На рисунке \ref{img:figure} приведена графическая интерпретация результата замеров времени работы реализаций алгоритмов для различных линейных размеров матриц.
\img{0.45}{figure}{Результаты замеров времени работы реализаций алгоритмов для различных линейных размеров матриц}
\clearpage

\section{Постановка эксперимента}

Автоматическая параметризация была проведена на двух классах данных --- \ref{class1} и \ref{class2}. Алгоритм был запущен для набора значений $\alpha, \rho \in (0, 1)$.

Итоговая таблица значений параметризации состоит из следующих колонок:
\begin{itemize}[label=---]
	\item $\alpha$ --- коэффициент жадности;
	\item $\rho$ --- коэффициент испарения;
	\item days --- количество дней жизни колонии муравьев;
	\item result --- эталонный результат, полученный методом полного перебора для проведения данного эксперимента;
	\item delta --- разность полученного основаным на муравьином алгоритме методом значения и эталонного значения на данных значениях параметров, показатель качества решения.
\end{itemize}

\textit{Цель эксперимента} --- определить комбинацию параметров, которые позволяют решать задачу наилучшим образом для выбранного класса данных. Качество решения зависит от количества дней и погрешности измерений.

\subsection{Класс данных 1}
\label{class1}

Класс данных 1 представляет собой матрицу смежности размером 9 элементов (небольшой разброс значений --- от 1 до 2), которая представлена далее.

\begin{equation}
	K_{1} = \begin{pmatrix}
		0 & 1 & 1 & 1 & 2 & 1 & 1 & 2 & 1 \\
		1 & 0 & 2 & 2 & 1 & 1 & 2 & 1 & 1  \\
		1 & 2 & 0 & 2 & 2 & 1 & 2 & 2 & 1 \\
		1 & 2 & 2 & 0 & 2 & 2 & 2 & 2 & 1 \\
		2 & 1 & 2 & 2 & 0 & 2 & 2 & 2 & 1  \\
		1 & 1 & 1 & 2 & 2 & 0 & 2 & 1 & 2 \\
		1 & 2 & 2 & 2 & 2 & 2 & 0 & 2 & 2 \\
		2 & 1 & 2 & 2 & 2 & 1 & 2 & 0 & 1 \\
		1 & 1 & 1 & 1 & 1 & 2 & 2 & 1 & 0 \\
	\end{pmatrix}
\end{equation}

Для данного класса данных приведена таблица \ref{table_kd1}	с выборкой параметров, которые наилучшим образом решают поставленную задачу, полные результаты параметризации приведены в приложении А.

В выборке, разделенной на подгруппы по признаку значения параметра $\alpha$, для пары ($\alpha$, $\rho$) выбран набор значений параметров, обеспечивающих наилучший результат приближения (наименьшее значение параметра delta). Если одинаковый результат параметра delta достигается для нескольких кортежей ($\alpha$, $\rho$, days, result, delta), содержащих одинаковые значения параметров ($\alpha$, $\rho$), среди них выбирается кортеж, содержащий наименьшее значение параметра days. 
\pagebreak
\begin{center}
	\label{table_kd1}
	\begin{longtable}[c]{|c|c|c|c|c|}
		\captionsetup{justification=raggedright,singlelinecheck=off}
		\caption{Выборка из параметров для класса данных 1}\\
		\hline
		$\alpha$ & $\rho$ & days & result & delta \\ \hline
		0.1 & 0.1 & 300 & 10 & 0 \\
		0.1 & 0.2 & 300 & 10 & 0 \\
		0.1 & 0.5 & 300 & 10 & 0 \\
		0.1 & 0.6 & 100 & 10 & 0 \\
		0.1 & 0.7 & 300 & 10 & 0 \\ \hline
		0.2 & 0.1 & 500 & 10 & 0 \\
		0.2 & 0.2 & 500 & 10 & 1 \\
		0.2 & 0.4 & 500 & 10 & 1 \\
		0.2 & 0.5 & 500 & 10 & 0 \\
		0.2 & 0.7 & 300 & 10 & 0 \\ \hline
		0.3 & 0.1 & 300 & 10 & 0 \\
		0.3 & 0.2 & 300 & 10 & 0 \\
		0.3 & 0.4 & 500 & 10 & 0 \\
		0.3 & 0.8 & 300 & 10 & 0 \\
		0.3 & 0.9 & 300 & 10 & 0 \\ \hline
		0.4 & 0.1 & 500 & 10 & 0 \\
		0.4 & 0.4 & 500 & 10 & 0 \\
		0.4 & 0.6 & 500 & 10 & 0 \\
		0.4 & 0.7 & 300 & 10 & 0 \\
		0.4 & 0.9 & 300 & 10 & 0 \\ \hline
		0.5 & 0.1 & 500 & 10 & 1 \\
		0.5 & 0.3 & 500 & 10 & 0 \\
		0.5 & 0.5 & 300 & 10 & 0 \\
		0.5 & 0.7 & 300 & 10 & 0 \\
		0.5 & 0.9 & 500 & 10 & 1 \\ \hline
		0.6 & 0.1 & 50 & 10 & 1 \\
		0.6 & 0.3 & 500 & 10 & 0 \\
		0.6 & 0.5 & 500 & 10 & 1 \\
		0.6 & 0.6 & 300 & 10 & 0 \\
		0.6 & 0.9 & 500 & 10 & 0 \\ \hline
		0.7 & 0.2 & 500 & 10 & 0 \\
		0.7 & 0.4 & 500 & 10 & 0 \\
		0.7 & 0.5 & 300 & 10 & 0 \\
		0.7 & 0.7 & 500 & 10 & 0 \\
		0.7 & 0.9 & 500 & 10 & 0 \\ \hline
		0.8 & 0.1 & 500 & 10 & 1 \\
		0.8 & 0.3 & 500 & 10 & 1 \\
		0.8 & 0.7 & 500 & 10 & 0 \\
		0.8 & 0.8 & 500 & 10 & 1 \\
		0.8 & 0.9 & 300 & 10 & 0 \\ \hline
		0.9 & 0.1 & 500 & 10 & 0 \\
		0.9 & 0.3 & 500 & 10 & 1 \\
		0.9 & 0.5 & 500 & 10 & 0\\
		0.9 & 0.7 & 500 & 10 & 1 \\
		0.9 & 0.9 & 500 & 10 & 1 \\ \hline
	\end{longtable}
\end{center}

\subsection{Класс данных 2}
\label{class2}


Класс данных 2 представляет собой матрицу смежности размером 9 элементов (большой разброс значений - от 1000 до 9999), которая представлена далее.

В выборке, разделенной на подгруппы по признаку значения параметра $\alpha$, для пары ($\alpha$, $\rho$) выбран набор значений параметров, обеспечивающих наилучший результат приближения (наименьшее значение параметра delta). Если одинаковый результат параметра delta достигается для нескольких кортежей ($\alpha$, $\rho$, days, result, delta), содержащих одинаковые значения параметров ($\alpha$, $\rho$), среди них выбирается кортеж, содержащий наименьшее значение параметра days. 

\begin{equation}
	K_{2} = \begin{pmatrix}
		0 & 3335 & 6874 & 6965 & 6380 & 9302 & 2182 & 2668 & 9645\\
		3335 & 0 & 2057 & 9364 & 3464 & 4552 & 6097 & 5318 & 4220\\
		6874 & 2057 & 0 & 2695 & 5333 & 8417 & 2209 & 4219 & 9177\\ 
		6965 & 9364 & 2695 & 0 & 1073 & 3715 & 1777 & 6458 & 1082\\
		6380 & 3464 & 5333 & 1073 & 0 & 3111 & 3677 & 5733 & 2078\\
		9302 & 4552 & 8417 & 3715 & 3111 & 0 & 8884 & 7863 & 3266\\
		2182 & 6097 & 2209 & 1777 & 3677 & 8884 & 0 & 7885 & 1221\\ 
		2668 & 5318 & 4219 & 6458 & 5733 & 7863 & 7885 & 0 & 9604\\ 
		9645 & 4220 & 9177 & 1082 & 2078 & 3266 & 1221 & 9604 & 0
	\end{pmatrix}
\end{equation}


Для данного класса данных приведена таблица \ref{table_kd2}	с выборкой параметров, которые наилучшим образом решают поставленную задачу, полные результаты параметризации приведены в приложении Б.

\begin{center}
	\label{table_kd2}
	\begin{longtable}[c]{|c|c|c|c|c|}
		\captionsetup{justification=raggedright,singlelinecheck=off}
		\caption{Выборка из параметров для класса данных 2}\\
		\hline		
		$\alpha$ & $\rho$ & days & result & delta \\ \hline
		0.1 & 0.1 & 100 & 22165 & 0 \\
		0.1 & 0.3 & 100 & 22165 & 0 \\
		0.1 & 0.6 & 100 & 22165 & 0 \\ \hline
		0.2 & 0.2 & 300 & 22165 & 0 \\
		0.2 & 0.6 & 300 & 22165 & 0 \\
		0.2 & 0.9 & 100 & 22165 & 0 \\ \hline
		0.3 & 0.3 & 300 & 22165 & 0\\
		0.3 & 0.6 & 300 & 22165 & 0 \\
		0.3 & 0.8 & 100 & 22165 & 0 \\ \hline
		0.4 & 0.2 & 100 & 22165 & 0 \\
		0.4 & 0.3 & 300 & 22165 & 0 \\
		0.4 & 0.5 & 300 & 22165 & 0 \\ \hline
		0.5 & 0.2 & 300 & 22165 & 0 \\
		0.5 & 0.4 & 300 & 22165 & 0 \\
		0.5 & 0.8 & 300 & 22165 & 0 \\ \hline
		0.6 & 0.1 & 500 & 22165 & 0 \\
		0.6 & 0.3 & 500 & 22165 & 0 \\
		0.6 & 0.8 & 500 & 22165 & 0 \\ \hline
		0.7 & 0.4 & 500 & 22165 & 0 \\
		0.7 & 0.7 & 500 & 22165 & 0 \\
		0.7 & 0.9 & 500 & 22165 & 1426 \\ \hline
		0.8 & 0.2 & 500 & 22165 & 0 \\
		0.8 & 0.4 & 500 & 22165 & 0 \\
		0.8 & 0.6 & 500 & 22165 & 0 \\ \hline
		0.9 & 0.1 & 500 & 22165 & 1426 \\
		0.9 & 0.3 & 500 & 22165 & 1561 \\
		0.9 & 0.7 & 500 & 22165 & 1561 \\ \hline
	\end{longtable}
\end{center}

\section*{Вывод}

В результате исследования было получено, что использование муравьиного алгоритма наиболее эффективно при больших размерах матриц. Так, при размере матрицы, равном 2, муравьиный алгоритм медленнее алгоритма полного перебора в 332 раза, а при размере матрицы, равном 9, муравьиный алгоритм быстрее алгоритма полного перебора в 12 раза. Из рисунка \ref{img:figure} следует, что при размерах матриц больше 7 следует использовать муравьиный алгоритм, но стоит учитывать, что он не гарантирует получения глобального минимума при решении задачи.

Также при проведении параметризации с классами данных было получено, что на первом классе данных муравьиный алгоритм лучше всего показывает себя при параметрах:
\begin{itemize}[label=---]
	\item $\alpha = 0.1, \rho = 0.1, 0.3, 0.6$;
	\item $\alpha = 0.2, \rho = 0.9$;
	\item $\alpha = 0.3, \rho = 0.8, 0.9$;
	\item $\alpha = 0.4, \rho = 0.7$;
	\item $\alpha = 0.6, \rho = 0.1$.
\end{itemize}  
Следовательно, для класса данных 1 рекомендуется использовать данные параметры. 

Для класса данных 2 было получено, что наилучшим образом алгоритм работает на значениях параметров, которые представлены далее:
\begin{itemize}[label=---]
	\item $\alpha = 0.1, \rho = 0.3, 0.3, 0.7$;
	\item $\alpha = 0.2, \rho = 0.9$;
	\item $\alpha = 0.3, \rho = 0.8$;
	\item $\alpha = 0.4, \rho = 0.5$;
	\item $\alpha = 0.5, \rho = 0.2, 0.4, 0.8$.
\end{itemize} 
Для второго класса данных 2 рекомендуется использовать данные параметры.

Также во время исследования было замечено, что число дней жизни колонии значительно влияет на качество решения: чем значение параметра $days$ больше, тем меньше отклонение решения от эталонного.


