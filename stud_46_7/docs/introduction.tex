% !TeX spellcheck = ru_RU
\chapter*{Введение}
\addcontentsline{toc}{chapter}{Введение}
Задача оптимизации неэффективных с точки зрения времени выполнения алгоритмов всегда были важны.
Одним из таких алгоритмов является полный перебор для решения задачи поиска оптимальных путей.


\textbf{Целью данной работы} является параметризация метода решения задачи коммивояжера на основе муравьиного метода.

\textbf{Задачи данной лабораторной работы:}
\begin{enumerate}
	\item описать задачу коммивояжера;
	\item описать методы решения задачи коммивояжера --- метод полного перебора и метод на основе муравьиного алгоритма;
	\item привести схемы муравьиного алгоритма и алгоритма, позволяющего решить задачу коммивояжера методом полного перебора;
	\item разработать и реализовать программный продукт, позволяющий решить задачу коммивояжера исследуемыми методами;
	\item сравнить по времени метод полного перебора и метод на основе муравьиного алгоритма;
	\item описать и обосновать полученные результаты в отчете о выполненной лабораторной работе.
\end{enumerate}
