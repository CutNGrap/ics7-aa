% !TeX spellcheck = ru_RU
\newpage
\chapter*{Введение}
\addcontentsline{toc}{chapter}{Введение}
Важной частью программирования является работа со строками. Часто возникает потребность в сравнении строк по длине и по содержимому. О таких алгоритмах и пойдет речь в данной работе.
\newline

\textbf{Целью данной лабораторной работы} является изучение, реализация и исследование алгоритмов нахождения расстояний Левенштейна и Дамерау-Левенштейна.\\
\textbf{Задачи данной лабораторной работы:}
\begin{enumerate}
%	список со скобкой вместо точки у цифры
	\item изучение расстояний Левенштейна и Дамерау-Левенштейна;
	\item разработка и реализация алгоритмов нахождения расстояния Левенштейна и Дамерау-Левенштейна;
	\item проведение сравнительного анализа по времени матричной, рекурсивной и рекурсивной с использованием кэша реализаций алгоритма нахождения расстояния Дамерау-Левенштейна;
	\item описание и обоснование полученных результатов в отчете о выполненной лабораторной работе.
\end{enumerate}
