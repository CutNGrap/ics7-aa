% !TeX spellcheck = ru_RU
\chapter*{Заключение}
\addcontentsline{toc}{chapter}{Заключение}

В результате исследования было определено, что время выполнения рекурсивного алгоритма поиска расстояния Дамерау-Левенштейна на длинах строке больше 5 многократно (более чем в 6 раз) превосходит время выполнения матричной реализации алгоритма и рекурсивного алгоритма с кэшем. Однако скорость роста затрат памяти рекурсивного алгоритма линейна, в то время как затраты памяти других двух алгоритмов растут как функция $f(n\cdot m)$. 

В результате исследования были получены следующие результаты замеров времени работы алгоритмов (для $n = m = 9$):
% поправить вылеты за край строки
\begin{itemize}
	\item 34 мс --- поиск расстояния Левенштейна;
	\item 42 мс ---  матричная реализация поиска расстояния Дамерау\hyp{}Левенштейна;
	\item 503 мс --- рекурсивная реализация поиска расстояния Дамерау\hyp{}Левенштейна;
	\item 75 мс --- рекурсивная реализация поиска расстояния Дамерау\hyp{}Левенштейна с кэшем.
\end{itemize}
 Для строк длины 9 время выполнения составляет 42 мс для матричного метода и 503 мс для рекурсивной функции (превосходство более чем в 11 раз)

Цель, поставленная в начале работы, была достигнута. Кроме того были достигнуты поставленные задачи:

\begin{itemize}
	\item были изучены расстояния Левенштейна и Дамерау-Левенштейна;
	\item были разработаны и реализованы алгоритмы нахождения расстояния Левенштейна и Дамерау-Левенштейна;
	\item был проведен сравнительный анализ по времени матричной, рекурсивной и рекурсивной с использованием кэша реализаций алгоритма нахождения расстояния Дамерау-Левенштейна;
	\item был подготовлен отчет о лабораторной работе.
\end{itemize}