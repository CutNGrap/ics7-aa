% !TeX spellcheck = ru_RU
\chapter{Технологическая часть}
В данном разделе будут рассмотрены средства реализации, а также представлены листинги алгоритмов определения расстояния Левенштейна и Даме- рау-Левенштейна.
\section{Средства реализации}
% вопрос к слову "минимальным затратами" - относительно чего? обоснование так себе. Быстрее аналогов --- так себе идея
% требования: обработка строк, рекурсия, проц время === язык подходит под требования === хорошо
Для данной работы был выбран язык Python \cite{python}. Для данной лабораторной работы требуются инструменты для работы со строками, замеров процессорного времени работы выполняемой программы, визуализации полученных данных. Все перечисленные инструменты присутствуют в выбранном языке программирования


\section{Листинги кода}
% бандитский код
В листингах \ref{leven} --- \ref{damercache} приведена реализация алгоритмов нахождения расстояния Левенштейна и Дамерау-Левенштейна.

\lstinputlisting[language=Python, firstline=1, lastline=13, label=leven,caption=Функция нахождения расстояния Левенштейна итеративно]{../src/func.py}
\lstinputlisting[language=Python, firstline=14, lastline=18, label=leven1,caption=Функция нахождения расстояния Левенштейна итеративно]{../src/func.py}
\lstinputlisting[language=Python, firstline=20, lastline=41, label=damerau,caption=Функция нахождения расстояния Дамерау-Левенштейна итеративно]{../src/func.py}
\pagebreak
\lstinputlisting[language=Python, firstline=43, lastline=53, label=damerrec,caption=Функция нахождения расстояния Дамерау-Левенштейна рекурсивно]{../src/func.py}
\lstinputlisting[language=Python, firstline=56, lastline=73, label=damercache,caption=Функция нахождения расстояния Дамерау-Левенштейна рекурсивно с использованием кэша]{../src/func.py}
\clearpage



\section{Функциональные тесты}

% таблица 3.1, не схема; нижняя строка таблицы отвалиалсь
В таблице \ref{tbl:functional_test} приведены тесты для функций, реализующих алгоритмы нахождения расстояния Левенштейна и Дамерау-Левенштейна.
\begin{table}[h]
	\begin{center}
		\begin{threeparttable}
			\captionsetup{justification=raggedright,singlelinecheck=off}
			\caption{\label{tbl:functional_test} Функциональные тесты}
			\begin{tabular}{|c|c|c|c|c|}
				\hline
				& & & \multicolumn{2}{c|}{Ожидаемый результат} \\
				\hline
				№&Строка 1&Строка 2&Левенштейн&Дамерау-Левенштейн \\
				\hline
				1&""&""&0&0 \\
				\hline
				2&""&teststr&7&7 \\
				\hline
				3&smthfortest&""&11&11 \\
				\hline
				4&test&nest&1&1\\
				\hline
				5&usage&sausage&2&2\\
				\hline
				6&phn&phone&2&2\\
				\hline
				7&ucnle&uncle&2&1\\
				\hline
				8&sure&user&3&2\\
				\hline
				9&plaanee&plane&2&2\\
				\hline
				10&labtpoe&laptop&3&3\\
				\hline
			\end{tabular}
		\end{threeparttable}
	\end{center}
\end{table}
\newline
Все тесты пройдены функциями успешно.
\section*{Вывод}
В данном разделе были разработаны и приведены исходные коды четырех алгоритмов, рассмотренных и описанных ранее. Также, были описаны выбранные средства реализации алгоритмов и функциональные тесты для описанных алгоритмов.