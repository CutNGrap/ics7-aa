% !TeX spellcheck = ru_RU
\chapter{Аналитическая часть}

В данном разделе будут разобраны алгоритмы нахождения расстояний Левенштейна и Дамерау-Левенштейна.


% разобраться с newline, красная строка то есть, то нет
\section{Расстояние Левенштейна}

\textbf{Расстояние Левенштейна} \cite{Levenshtein} между двумя строками --- величина, определяемая как минимальное количество редакционных изменений, необходимых для преобразования одной строки в другую. \newline
Выделяют следующие типы редакционных преобразований:
\begin{itemize}
	\item I (Insert) --- вставка символа в произвольной позиции;
	\item D (Delete) --- удаление символа в произвольной позиции;
	\item R (Replace) --- замена символа на другой;
\end{itemize}

% Отсутствие редакц изменений....
	M (Match) --- совпадение двух символов (цена --- 0). \newline

Во время нахождения минимального количества редакционных изменений возникает проблема взаимного выравнивания строк. Рассмотрим пример для строк "развлечения" и "увлечение", он представлен на рис. \ref{table_1}.\newline
\pagebreak
%\renewcommand{\tablename}{}
\begin{table}[!h]
	\begin{tabular}{ c|c|c|c|c|c|c|c|c|c|cl }
		р & а & з & в & л & е & ч & е & н & и & я & \rdelim\}{4}{*}[10 редакционных изменений] \\
		у & в & л & е & ч & е & н & и & е & ~ & ~ \\
		R & R & R & R & R & M & R & R & R & D & D \\
		1 & 1 & 1 & 1 & 1 & 0 & 1 & 1 & 1 & 1 & 1 
	\end{tabular}
%\centering
\end{table}
\begin{table}[!h]
	\begin{tabular}{ c|c|c|c|c|c|c|c|c|c|cl }
		р & а & з & в & л & е & ч & е & н & и & я & \rdelim\}{4}{*}[4 редакционных изменения]\\
		~ & ~ & у & в & л & е & ч & е & н & и & е \\
		D & D & R & M & M & M & M & M & M & M & R \\
		1 & 1 & 1 & 0 & 0 & 0 & 0 & 0 & 0 & 0 & 1 
	\end{tabular}
	\caption{}
	\label{table_1}
%\centering
\end{table}

Чтобы снять проблему выравнивания используют рекуррентную формулу (\ref{eq:D}).
Пусть даны строки $S_1$ и $S_2$ длины $N$ и $M$ соответственно.
Тогда $S_1[1 \ldots i]$ --- подстрока $S_1$ длины i, начиная с первого символа.
Аналогично, $S_2[1 \ldots j]$ --- подстрока $S_2$ длины j, начиная с первого символа.\newline

Расстояние Левенштейна между подстроками $S_1[1 \ldots i]$ и $S_2[1 \ldots j]$ находится по формуле \ref{eq:D}
\begin{equation}
	\label{eq:D}
	D(i, j) = \begin{cases}
		0, &\text{i = 0, j = 0}\\
		i, &\text{j = 0, i > 0}\\
		j, &\text{i = 0, j > 0}\\
		\min ( \\
		\qquad D(i, j-1) + 1,\\
		\qquad D(i-1, j) + 1, \\
		\qquad D(i-1, j-1) + m(i, j)\\
		),&\text{иначе}
	\end{cases}
\end{equation}
где функция $m(i, j)$ определяется по формуле
\begin{equation}
	\label{eq:m}
	m(i, j) = \begin{cases}
		0, &\text{$ S_1[i] = S_2[j] $}\\
		1, &\text{иначе}\\
	\end{cases}
\end{equation}
%маленькие подпункты без нумерации страниц, subsection == более 1 страницы
%\subsection{Рекурсивный алгоритм нахождения расстояния}

Рекурсивный алгоритм реализует формулу (\ref{eq:D}).
Реализация данной формулы может быть неэффективна по времени выполнения, особенно при больших $N$ и $M$, так как множество промежуточных значений $D(i, j)$ вычисляется повторно.  

%\subsection{Матричный алгоритм нахождения расстояния}
Альтернативным решением является использование матрицы для хранения расстояний $D(i, j)$.
Тогда алгоритм нахождения расстояния Левенштейна представляет собой построчное заполнение матрицы $A_{(N + 1)(M + 1)}$ значениями $D(i, j)$.
Первая строка и первый столбец матрицы заполняются по формуле
\begin{equation}
	\label{eq:fir}
	A[i][j] = \begin{cases}
		j, & i = 0\\
		i, & j = 0
	\end{cases}
\end{equation}
Остальные строки и столбцы заполняются по формуле
\begin{equation}
	\label{eq:oth}
	A[i][j] = min \begin{cases}
		A[i-1][j] + 1\\
		A[i][j-1] + 1\\
		A[i-1][j-1] + m(i, j)\\
	\end{cases}
\end{equation}
Результат вычисления расстояния Левенштейна будет находится в ячейке матрицы $A[N, M]$.
	
\section{Расстояние Дамерау-Левенштейна}

\textbf{Расстояние Дамерау-Левенштейна} \cite{damerau-levenshtein} между двумя строками, состоящими из конечного числа символов — это минимальное число редакционных изменений, необходимых для перевода одной строки в другую.
Является модификацией расстояния Левенштейна --- к списку редакционных изменений добавлена операция \textit{транспозиции}, то есть перестановки двух соседних символов.

Расстояние Дамерау-Левенштейна может быть найдено по формуле
\clearpage
\begin{equation}
	\label{eq:D1}
	D(i, j) = \begin{cases}
		\max(i, j), &\text{если }\min(i, j) = 0,\\
		\min ( \\
		\qquad d_{a,b}(i, j-1) + 1,\\
		\qquad d_{a,b}(i-1, j) + 1,\\
		\qquad d_{a,b}(i-1, j-1) + m(i, j), &\text{иначе}\\
		\qquad \left[ \begin{array}{cc}d_{a,b}(i-2, j-2) + 1, &\text{если }i,j > 1;\\
			\qquad &\text{}a[i] = b[j-1]; \\
			\qquad &\text{}b[j] = a[i-1]\\
			\qquad \infty, & \text{иначе}\end{array}\right.\\
		)
	\end{cases},
\end{equation}
Формула опирается на те же соображения, что и формула  (\ref{eq:D}), с учетом добавления операции транспозиции.

%\subsection{Рекурсивный алгоритм нахождения расстояния}
Рекурсивный алгоритм вычисления расстояния Дамерау-Левенштейна реализует формулу (\ref{eq:D1}).
Однако, как и в рекурсивном алгоритме нахождения расстояния Левенштейна, в данном алгоритме происходит многократное вычисление промежуточных $D(i, j)$.
В качестве оптимизации можно использовать матрицу для хранения уже вычисленных промежуточных $D(i, j)$.

%\subsection{Матричный алгоритм нахождения расстояния}
Матричный алгоритм  нахождения расстояния Дамерау-Левенштейна представляет собой построчное заполнение матрицы $A_{(N + 1)(M +1 )}$ значениями $D(i, j)$. Результат вычисления расстояния Дамерау-Левенштейна будет находится в ячейке матрицы $A[N, M]$.
Первая строка и первый столбец матрицы заполняются по формуле
\begin{equation}
	\label{eq:first}
	A[i][j] = \begin{cases}
	j, & i = 0\\
	i, & j = 0
	\end{cases}
\end{equation}
Остальные строки и столбцы заполняются по формуле
\begin{equation}
	\label{eq:other}
	A[i][j] = min \begin{pmatrix}
		A[i-1][j] + 1, \\
		A[i][j-1] + 1, \\
		A[i-1][j-1] + m(i, j),\\
		t(i, j)
	\end{pmatrix}
\end{equation}
где функция $t(i, j)$ определяется по формуле
\begin{equation}
	\label{eq:t}
	t(i, j) =\left[ \begin{array}{cc}A[i-2][j-2] + 1, \text{ если $i > 1$, $j > 1$},
	&S_1[i - 2] = S_2[j - 1], \\
	&S_2[i - 1] = S_2[j - 2]\\
	\infty, & \text{иначе}\end{array}\right.\\
\end{equation}

%\subsection{Рекурсивный алгоритм нахождения расстояния с использованием кэша}
Для сокращения количества вычислений промежуточных значений D(i, j) в рекурсивной реализации алгоритма, можно использовать кэш, представляющий собой матрицу.  
Если в матрице уже есть значение текущего D(i, j), это расстояние не рассчитывается снова, а берется из матрицы. Иначе происходит расчет текущего значения, которое заносится в матрицу.
\section*{Вывод}
В данном разделе были рассмотрены алгоритмы нахождения расстояния Левенштейна и Дамерау-Левенштейна. Формулы нахождения данных расстояний являются рекуррентными, что позволяет реализовывать алгоритмы как рекурсивно, так и итеративно.