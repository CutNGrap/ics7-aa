% !TeX spellcheck = ru_RU
\newpage
\chapter*{Введение}
\addcontentsline{toc}{chapter}{Введение}
Одной из важнейших процедур обработки структурированной информации является сортировка. Сортировкой называют процесс перегруппировки заданной последовательности (кортежа) объектов в некотором определенном порядке. 

Определенный порядок (например, упорядочение в алфавитном порядке, по возрастанию или убыванию количественных характеристик, по классам, типам и т.п.) в последовательности объектов необходим для более эффективной работы с этой последовательностью. В частности, одной из целей сортировки является облегчение последующего поиска элементов в отсортированном массиве. 

Существует множество различных методов сортировки данных. Однако любой алгоритм сортировки можно разбить на три основные части:
\begin{enumerate}
	\item сравнение, определяющее упорядоченность пары элементов;
	\item перестановка, меняющая местами пару элементов;
	\item сортирующий алгоритм, который осуществляет сравнение и перестановку элементов данных до тех пор, пока все эти элементы не будут упорядочены.
\end{enumerate}

\textbf{Цель лабораторной работы:} исследование трудоемкости алгоритмов сортировок: пузырьком, пирамидальной, блочной.

\textbf{Задачи лабораторной работы:}
\begin{itemize}
	\item изучить и реализовать 3 алгоритма сортировки: пузырьком, пирамидальная, блочная;
	\item провести сравнительный анализ трудоемкости алгоритмов на основе теоретических расчетов;
	\newline
	\item провести сравнительный анализ алгоритмов на основе экспериментальных данных;
	\item составить отчет о выполненной лабораторной работе.
\end{itemize}