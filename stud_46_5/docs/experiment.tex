% !TeX spellcheck = ru_RU
\chapter{Исследовательская часть}

В данном разделе  приведены примеры работы программы, а также проведен сравнительный анализ последовательной и параллельной реализаций при различном количестве заявок.

\section{Технические характеристики}

Ниже приведены технические характеристики устройства, на котором было проведено измерение времени работы ПО:

\begin{itemize}
	\item операционная система Windows 10 Домашняя Версия 21H1 \cite{windows} x86\_64;
	\item оперативная память 8 Гбайт 2133 МГц;
	\item процессор Intel Core i5-8300H с тактовой частотой 2.30 ГГц \cite{intel}, 4 физических ядра, 8 логических ядер.
\end{itemize}

\section{Демонстрация работы программы}

На рисунке \ref{img:example} представлен приме работы программы. Пользователь, указывая соответствующие пункты меню, запускает последовательную или параллельную обработку заявок. Результаты работы на экран не выводятся, а записываются в файл $file.txt$. Пример результатов работы программы представлен на рисунке \ref{img:res}. Лог программы также на экран не выводится --- он записывается в файл. На рисунке \ref{img:log} представлен пример лог--файла.

\img{0.8}{example}{Пример работы программы}
\img{1}{res}{Пример результатов работы программы}
\img{0.7}{log}{Пример лог--файла}
\clearpage
\section{Время выполнения реализации алгоритма}

Для замеров времени использовалась функция получения значения системных часов $clock\_gettime()$ \cite{gettime}. Функция применялась два раза --- в начале и в конце измерения времени, значения полученных временных меток вычитались друг из друга для получения времени выполнения программы.

Документы заполнялись случайными словами из букв латинского алфавита. Для замеров времени использовались наборы из 30 документов по 20 слов каждый.
Замеры проводились по 100 раз для набора значений количества заявок {10, 20, 30, 40, 50, 60, 70, 80, 90, 100}.
В таблице \ref{tbl:threads} представлены замеры времени выполнения реализаций в зависимости от количества заявок.


\begin{table}[h]
	\begin{center}
		\begin{threeparttable}
			\caption{Результаты нагрузочного тестирования (в мс)}
			\label{tbl:threads}
			\begin{tabular}{|c|c|c|}
				\hline
				{Кол-во заявок} & {Последовательная работа} &{Параллельная работа} \\
				\hline
				 10 &32&8\\
				\hline
				 20 &65&16\\
				 \hline
				 30 &98&24\\
				 \hline
				 40 &131&32\\
				 \hline
				 50 &166&39\\
				 \hline
				 60 &198&49\\
				 \hline
				 70 &233&56\\
				 \hline
				 80 &267&64\\
				 \hline
				 90 &301&73\\
				 \hline
				 100 &332&81\\
				\hline
				
			\end{tabular}
		\end{threeparttable}
	\end{center}
\end{table}

\clearpage
На рисунке \ref{img:gr} приведена графическая интерпретация результатов замеров.

\img{1}{gr}{Результаты замеров времени работы реализаций алгоритма с разным количеством потоков в зависимости от количества документов}



\section*{Вывод}

В результате эксперимента было получено, что использование конвейерной обработки способно сократить время обработки 100 заявок в 4.1 раза.
В силу линейности графиков (рис. \ref{img:gr}) можно сказать, что на достаточно большом количестве заявок выигрыш параллельной обработки над последовательной во времени в абсолютных единицах будет увеличиваться.


