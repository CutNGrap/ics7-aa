% !TeX spellcheck = ru_RU
\chapter{Аналитическая часть}

В данном разделе рассмотрена конвейерная обработка данных, представлена информация об исследуемом алгоритме выделения терминов для классификации.

%секшн
\section{Конвейерная обработка данных}

Конвейер \cite{conway} (англ. conway) — организация вычислений, при которой увеличивается количество выполняемых инструкций за единицу времени за счет использования принципов параллельности.
Конвейерная обработка в общем случае основана на разделении подлежащей исполнению функции на более мелкие части, называемые ступенями, и выделении для каждой из них отдельного блока аппаратуры. 
Так, обработку любой машинной команды можно разделить на несколько этапов, организовав передачу данных от одного этапа к следующему. При этом конвейерную обработку можно использовать для совмещения этапов выполнения разных команд. Производительность при этом возрастает, благодаря тому, что одновременно на различных ступенях конвейера выполняется несколько команд. 
Конвейерная обработка такого рода широко применяется во всех современных быстродействующих процессорах. 

Конвейеризация позволяет увеличить пропускную способность процессора (количество команд, завершающихся в единицу времени), но она не сокращает время выполнения отдельной команды. В действительности она даже несколько увеличивает время выполнения каждой команды из-за накладных расходов, связанных с хранением промежуточных результатов. 
Однако увеличение пропускной способности означает, что программа будет выполняться быстрее по сравнению с простой, неконвейерной схемой.

%\section{Документная частота}
Самая простая и вполне эффективная \cite{DF} техника оценки «важности терминов для классификации» основана на наблюдении того, что значительное число терминов коллекции встречаются в малом числе документов, а наибольшую информативность имеют термины со средней или даже высокой частотой, если предварительно были удалены стоп-слова. Данная техника может применяться как единственная, так и предшествовать другой технике отбора признаков.

Таким образом, алгоритм выделения наиболее информативных терминов (из выборки документов) состоит на подсчете документной частоты (DF) --- количества документов, в которых встречается термин $t_k$ и выбором терминов с наибольшими значениями DF.

%\section{Описание этапов обработки данных конвейером}

В качестве операций, выполняющихся на конвейере, взяты следующие:

\begin{enumerate}
	\item создание заданного количества текстовых документов для последующей обработки;
	\item обсчет документной частоты DF;
	\item вывод наиболее часто употребляемых терминов в новый файл.
\end{enumerate}

Ленты конвейера будут передавать друг другу заявки. 
Первый этап будет формировать заявку, которая будет передаваться от этапа к этапу.

Заявка должна содержать:

\begin{itemize}
	\item матрицу строк, представляющую собой набор документов;
	\item словарь, содержащий документную частоту для каждого обнаруженного слова;
	\item временные отметки.
\end{itemize}


\section*{Вывод}
В данном разделе была теоретически разобрана конвейерная обработка данных, представлена информация об исследуемом алгоритме выделения терминов для классификации.