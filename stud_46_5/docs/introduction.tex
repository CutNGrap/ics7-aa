% !TeX spellcheck = ru_RU
\chapter*{Введение}
\addcontentsline{toc}{chapter}{Введение}

Конвейерная обработка позволяет ускорить обработку данных благодаря использованию принципов параллельности. Идея взята из работы реальных конвейерных лент — данные поступают на обработку и проходят несколько независимых друг от друга стадий, что позволяет одновременно обрабатывать несколько единиц данных.

\textbf{Целью данной работы} является применение принципов конвейерной обработки данных.

\textbf{Задачи данной лабораторной работы:}
\begin{enumerate}
	\item описание основ конвейерной обработки данных;
	\item реализация алгоритма выделения наиболее информативных терминов набора документов на основе документной частоты;
	\item разработка программного продукта, позволяющего выполнять как последовательную, так и конвейерную реализацию озвученного алгоритма;
	\item проведение сравнительного анализа временной эффективности линейной и параллельной конвейерной реализаций алгоритма;
	\item описание и обоснование полученных результатов в отчете о выполненной лабораторной работе.
\end{enumerate}
