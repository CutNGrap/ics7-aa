% !TeX spellcheck = ru_RU

\chapter{Конструкторская часть}
В данном разделе разработаны последовательная и параллельная реализации работы стадий конвейера.
\section{Разработка алгоритмов}

На рисунке \ref{img:seq} представлен последовательный алгоритм работы стадий конвейера.

\img{0.7}{seq}{Схема последовательной работы алгоритма}
\clearpage

Параллельная работа будет реализована с помощью потоков, где каждый поток отвечает за свою стадию обработки.
Данные от потока к потоку будут передаваться через очереди.
Так как потоков, как и стадий обработки, три, то необходимо две очереди между потоками.

Для получения доступа к изменению содержимого очереди необходимо использовать примитив синхронизации --- мьютекс.
В противном случае, данные в очереди могут исказится.
Так как очереди две, то и мьютексов будет два.


На рисунке \ref{img:master} представлена схема главного потока при параллельной работе стадий конвейера.
\img{0.9}{master}{Схема параллельной работы}
\clearpage
На рисунках \ref{img:slave1} -- \ref{img:slave3} представлены схемы алгоритмов каждого из обработчиков (потоков) при параллельной работе.
\img{1.1}{slave1}{Схема алгоритма потока 1}
\clearpage
\img{0.9}{slave2}{Схема алгоритма потока 2}
\clearpage
\img{0.7}{slave3}{Схема алгоритма потока 3}
\section*{Вывод}
В данном разделе разработаны схемы последовательной и параллельной работы стадий конвейера.