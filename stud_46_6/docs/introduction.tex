% !TeX spellcheck = ru_RU
\chapter*{Введение}
\addcontentsline{toc}{chapter}{Введение}
По мере развития компьютерных систем стало понятно, что алгоритмы полного перебора неэффективны по времени.
Это вызвало необходимость создать новые алгоритмы, которые решают поставленную задачу на порядок быстрее стандартного решения прямого обхода.
В том числе это касается и словарей, в которых одной из основных операций является операция поиска.

\textbf{Целью данной работы} является получение навыка разработки метода поиска по словарю при ограничении на значение признака, заданном при помощи лингвистической переменной.

\textbf{Задачи данной лабораторной работы:}
\begin{enumerate}
	\item формализация объекта и его признака;
	\item проведение анкетирования респондентов;
	\item построения функции принадлежности термам числовых значений признака, описываемого лингвистической переменной, на основе статистической обработки мнений респондентов, выступающих в роли экспертов;
	\item описание и реализация алгоритма поиска в словаре объектов;
	\item описание и обоснование результатов в виде отчета о выполненной лабораторной работе.
\end{enumerate}
