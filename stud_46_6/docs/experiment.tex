% !TeX spellcheck = ru_RU
\chapter{Исследовательская часть}
В данном разделе приведены примеры работы программы, также описан проведенный эксперимент с лингвистической переменной.


%\section{Анкетирование респондентов}
Было проведено анкетирование следующих респондентов:
\begin{enumerate}
	\item Авсюнин, группа ИУ7-56Б --- Респондент 1;
	\item Ратников, группа ИУ7-52Б --- Респондент 2;
	\item Калашков, группа ИУ7-56Б --- Респондент 3;
	\item Комаров, группа ИУ7-52Б --- Респондент 4;
	\item Солопов, группа ИУ7-56Б --- Респондент 5;
	\item Чепрасов, группа ИУ7-56Б --- Респондент 6.
\end{enumerate}

В ходе анкетирования каждый из респондентов, выступая в качестве эксперта, указал промежутки, которые, по его мнению, определяет значение соответствующих термов.

Результаты анкетирования представлены в таблице~\ref{anket}. 

\begin{table}[h!]
	\begin{center}
		\begin{threeparttable}
			\captionsetup{justification=raggedright,singlelinecheck=off}
			\caption{Результаты анкетирования}
			\label{anket}
			\begin{tabular}{|c|c|c|c|c|c|c|}
				\hline
				& \multicolumn{6}{c|}{Респондент} \\
				\hline
				Терм & 1 & 2 & 3 & 4 & 5 & 6 \\
				\hline
				супер-цена &[1, 3]&[1, 2]&[1, 4]&[1, 5]&[1, 4]&[1, 2] \\
				\hline
				дешево & [4, 5] &  [3, 6] &[5, 7]&[6, 9]&[5, 6]&[3, 5] \\
				\hline
				средняя цена &[6, 9]&[7, 10]&[8, 10]&[10,12]&[9, 10]&[8, 12] \\
				\hline		
				не очень дорого &[10, 14]&[11, 14]&[11, 13]&[13, 15]&[11, 14]&[13, 15] \\
				\hline		
				дорого &[15, 17]&[15, 17]&[14, 16]&[16, 18]&[15, 16]&[17, 21] \\
				\hline		
				очень дорого &[18, 20]&[18, 19]&[17, 18]&[19, 20]&[15, 16]&[16, 18] \\
				\hline		
				неподъемная цена &  [21, 21] &[20, 21] &[19, 21]&[21, 21]&[17, 21]&[19, 21] \\
				\hline
			\end{tabular}
		\end{threeparttable}	
	\end{center}
\end{table}

\section{Построение функции принадлежности термам}

Для определения принадлежности числовых значений каждому из термов необходимо для каждого значения $k$ из рассматриваемого множества цен смартфона и для каждого терма найти количество респондентов, согласно которым значение из $k$ удовлетворяет сопоставляемому терму.
Данное значение необходимо разделить на общее количество респондентов --- это и будет значением функции $\mu$ для терма в точке.
Графики функций принадлежности числовых значений роста термам, приведен на рисунке \ref{img:figure}.

\img{0.6}{figure}{Принадлежность числовых значений термам}

В соответствии с полученными графиками функций принадлежности цен термам цены смартфона распределяются по термам следующим образом:
\begin{enumerate}
	\item <<супер-цена>> --- от 1 (десятка тысяч рублей) до 4 включительно;
	\item <<дешево>> --- от 5 до 6 включительно;
	\item <<средняя цена>> --- от 7 до 10 включительно;
	\item <<не очень дорого>> --- от 11 до 13 включительно;
	\item <<дорого>> --- от 14 до 16 включительно;
	\item <<очень дорого>> --- от 17 до 18 включительно;
	\item <<неподъемная цена>> --- от 19 до 21 включительно.
\end{enumerate}

%\section{Демонстрация работы программы}

На рисунке \ref{img:example} представлен пример работы программы. Пользователь, указывая желаемый запрос, получает ответ в виде таблицы на экране.

\img{0.8}{example}{Пример работы программы}

\section*{Вывод}

В данном разделе были приведены примеры работы программы, а также описан проведенный эксперимент с лингвистической переменной.


