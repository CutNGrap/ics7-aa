% !TeX spellcheck = ru_RU
\chapter{Аналитическая часть}

В данном разделе представлена информация о словаре как структуре данных и об алгоритме полного перебора.

%\section{Метод поиска по словарю при ограничении на значение признака, заданном при помощи лингвистической переменной}
Ниже приведены этапы метода поиска по словарю при ограничении на значение признака, заданном при помощи лингвистической переменной.

1. Разбить строку на токены.

2. Проверить запрос на корректность (проверить наличие ключевых слов <<смартфон>> и <<цена>> в списке токенов).

3. Используя список термов, выделить терм из входного запроса.

4. На основе входного терма определить диапазон искомых значений.

5. Сформировать выборку из словаря, удовлетворяющую диапазону искомых значений и подготовить ее к выводу.

%\section{Формализация объекта и его признака}
Объект <<смартфон>> формализуется следующим набором данных:
\begin{enumerate}
	\item серийный номер --- строка;
	\item номер паспорта владельца --- строка;
	\item название производителя --- строка;
	\item цена смартфона --- число.
\end{enumerate}

В качестве признака, по которому будет осуществляться выборка, использована цена смартфона  --- целое число, выражающее количество десятков тысяч рублей.


Следующие термы соответствуют признаку <<цена смартфона>>:
\begin{enumerate}
	\item <<супер-цена>>;
	\item <<дешево>>;
	\item <<средняя цена>>;
	\item <<не очень дорого>>;
	\item <<дорого>>;
	\item <<очень дорого>>;
	\item <<неподъемная цена>>.
\end{enumerate}

Цена смартфона будет рассматриваться в интервале от 1 (десятка тысяч рублей) до 21 (десятка тысяч рублей).

Для сопоставления каждому терму диапазона значений признака требуется определить значения функции принадлежности каждой величины признака каджому терму по материалам анкетирования экспертов. Одно значение функции определяется как отношение количества голосов респондентов за то, что у величины признака $x_j$ имеются свойства терма $t_i$, к количеству респондентов.

%\section{Словарь как структура данных}

Словарь \cite{dict} --- абстрактный тип данных (интерфейс к хранилищу данных), позволяющий хранить пары вида «(ключ, значение)» . Словарь поддерживает следующие операции: 
\begin{enumerate}
	\item \textit{insert(k, v)} --- добавление пары;
	\item \textit{find(k)} --- поиска пары по ключу;
	\item \textit{remove(k)} --- удаление пары по ключу.
\end{enumerate}

Словарь удобно рассматривать как массив, индексами которого могут выступать не только целые числа, но и другие типы данных. Например, часто в качестве ключа словаря используются строки 

%\section{Алгоритм полного перебора}
Алгоритм полного перебора \cite{AI} --- метод решения задачи, при котором по очереди рассматриваются все возможные варианты. 
В рамках данной лабораторной работы это означает перебор всех ключей словаря до тех пор, пока не будет найден искомый.

Трудоемкость алгоритма зависит от факта присутствия искомого ключа в словаре и (при условии, что ключ существует в словаре) его удаления от начала словаря. Например, при поиске ключа, находящегося в словаре первым, результат будет получен после первого же сравнения. Для ключа, расположенного в словаре вторым будет произведено два сравнения и так далее. Если принять трудоемкость одного сравнения за $C$, то трудоемкость нахождения ключа, расположенного на $n$-ой позиции будет равна $Cn$.

Если искомый ключ отсутствует в словаре, этот факт определится только после полного перебора всех ключей словаря, т.е. трудоемкость в таком случае равна трудоемкости для случая, когда ключ находится на последней позиции.

Если считать количество сравнений при поиске ключа в словаре дискретной случайно величиной, трудоемкость в среднем случае определяется с помощью математического ожидания данной случайной величины
\begin{equation}
	\label{mx}
	\begin{aligned}
		f_A = C \cdot MX = C \cdot \sum_{i=1}^N i \cdot \frac{1}{N} = C \cdot \frac{1 + N}{2},
	\end{aligned}
\end{equation}
где $f_A$ --- трудоемкость в среднем, MX --- математическое ожидание случайной величины, описывающей количество сравнений, N --- количество ключей словаря.
\section*{Вывод}
В данном разделе была представлена информация о словаре как структуре данных и об алгоритме полного перебора.