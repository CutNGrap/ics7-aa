% !TeX spellcheck = ru_RU
\newpage
\chapter*{Введение}
\addcontentsline{toc}{chapter}{Введение}
Матричная алгебра имеет обширные применения в различных отраслях знания – в математике, физике, информатике, экономике. Например, матрицы используется для решения систем алгебраических и дифференциальных уравнений, нахождения значений физических величин в квантовой теории, шифрования сообщений в Интернете.

Важной стороной работы с матрицами в программировании является оптимизация матричных операций (умножение, сложение, транспозиция и так далее), так как во многих задачах размеры матриц могут достигать больших значений. 
В данной лабораторной работе пойдет речь об оптимизации операции умножения матриц.


\textbf{Целью данной лабораторной работы} является изучение, реализация и исследование алгоритмов умножения матриц --- классический алгоритм, алгоритм Винограда, оптимизированный алгоритм Винограда.\\

\textbf{Задачи данной лабораторной работы:}
\begin{enumerate}
	\item изучение и реализация алгоритмов умножения матриц --- классического, Винограда и его оптимизацию;
	\item проведение сравнительного анализа по времени работы классического алгоритма и алгоритма Винограда;
	\item проведение сравнительного анализа по времени работы алгоритма Винограда и его оптимизации;
	\item описание и обоснование полученных результатов в отчете о выполненной лабораторной работе.
\end{enumerate}
