% !TeX spellcheck = ru_RU
\chapter{Технологическая часть}
В данном разделе рассмотрены средства реализации, а также представлены листинги реализаций алгоритмов умножения матриц --- классического, алгоритма Винограда и его оптимизации.
\section{Средства реализации}
Для данной работы был выбран язык Python \cite{python}. Для данной лабораторной работы требуются инструменты для работы с массивами, замеров процессорного времени работы выполняемой программы, визуализации полученных данных. Все перечисленные инструменты присутствуют в выбранном языке программирования


\section{Разработка алгоритмов}
В листингах \ref{lst:std} --- \ref{lst:win_o} приведены реализации алгоритмов умножения матриц.

\lst{3}{14}{std}{Функция умножения матриц по классическому алгоритму}
\lst{20}{48}{win}{Функция умножения матриц по алгоритму Винограда}
\clearpage
\lst{50}{87}{win_o}{Функция умножения матриц по оптимизированному алгоритму Винограда}
\clearpage



\section{Функциональные тесты}

В таблице \ref{tbl:functional_test} приведены тесты для функций, реализующих алгоритмы умножения матриц.
\begin{table}[h]
	\begin{center}
		\begin{threeparttable}
			\captionsetup{singlelinecheck=off}
			\caption{\label{tbl:functional_test} Функциональные тесты}
			\begin{tabular}{|c@{\hspace{7mm}}|c@{\hspace{7mm}}|c@{\hspace{7mm}}|c@{\hspace{7mm}}|c@{\hspace{7mm}}|c@{\hspace{7mm}}|}
				\hline
				Матрица 1 & Матрица 2 & Ожидаемый результат \\ 
				\hline
				
				$\begin{pmatrix}
					1 & 5 & 7\\
					2 & 6 & 8\\
					3 & 7 & 9
				\end{pmatrix}$ &
				$\begin{pmatrix}
					&
				\end{pmatrix}$ &
				Сообщение об ошибке \\ \hline
				
				$\begin{pmatrix}
					1 & 2 & 3
				\end{pmatrix}$ &
				$\begin{pmatrix}
					1 & 2 & 3
				\end{pmatrix}$ &
				Сообщение об ошибке \\ \hline
				
				$\begin{pmatrix}
					1 \\
					1 \\
					1
				\end{pmatrix}$ &
				$\begin{pmatrix}
					1 & 1 & 1
				\end{pmatrix}$ &
				$\begin{pmatrix}
					1 & 1 & 1\\
					1 & 1 & 1 \\
					1 & 1 & 1
				\end{pmatrix}$ \\ \hline
				
				$\begin{pmatrix}
					1 & 2 & 3 \\
					4 & 5 & 6 \\
					7 & 8 & 9
				\end{pmatrix}$ &
				$\begin{pmatrix}
					1 & 0 & 0 \\
					0 & 1 & 0 \\
					0 & 0 & 1
				\end{pmatrix}$ &
				$\begin{pmatrix}
					1 & 2 & 3 \\
					4 & 5 & 6 \\
					7 & 8 & 9
				\end{pmatrix}$ \\ \hline
				
				$\begin{pmatrix}
					7
				\end{pmatrix}$ &
				$\begin{pmatrix}
					8
				\end{pmatrix}$ &
				$\begin{pmatrix}
					56
				\end{pmatrix}$ \\ \hline
				
			\end{tabular}
		\end{threeparttable}
	\end{center}
\end{table}
\newline
Все тесты пройдены функциями успешно.
\section*{Вывод}
В данном разделе были разработаны и приведены исходные коды трех алгоритмов, рассмотренных и описанных ранее. Также, были описаны выбранные средства реализации алгоритмов и функциональные тесты для описанных алгоритмов.