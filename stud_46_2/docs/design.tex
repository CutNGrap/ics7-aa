% !TeX spellcheck = ru_RU

\chapter{Конструкторская часть}
В данном разделе разработаны схемы исследуемых алгоритмов, оценены трудоемкости в лучших и худших случаях.

\section{Разработка алгоритмов}
В данной части будут рассмотрены схемы классического алгоритма умножения матриц, алгоритма Винограда и его оптимизации. 
На рисунках \ref{img:std} -- \ref{img:win_o2} представлены схемы рассматриваемых алгоритмов.
\img{0.8}{std}{Схема классического алгоритма умножения матриц}
\img{0.8}{win_1}{Схема алгоритма Винограда умножения матриц}
\img{0.8}{win_2}{Схема алгоритма Винограда умножения матриц}
\img{0.8}{win_o1}{Схема оптимизированного алгоритма Винограда умножения матриц}
\img{0.8}{win_o2}{Схема оптимизированного алгоритма Винограда умножения матриц}
\clearpage

\section{Модель вычислений}

Для последующего вычисления трудоемкости введем модель вычислений \cite{model}:

\begin{enumerate}
	\item Операции из списка (\ref{for:opers}) имеют трудоемкость 1.
	\begin{equation}
		\label{for:opers}
		+, -, /, \%, ==, !=, <, >, <=, >=, [], ++, {-}-
	\end{equation}
	\item Трудоемкость оператора выбора \textit{if условие then A else B} рассчитывается, как (\ref{for:if}).
	\begin{equation}
		\label{for:if}
		f_{if} = f_{\text{условия}} +
		\begin{cases}
			f_A, & \text{если условие выполняется,}\\
			f_B, & \text{иначе.}
		\end{cases}
	\end{equation}
	\item Трудоемкость цикла рассчитывается, как (\ref{for:for}).
	\begin{equation}
		\label{for:for}
		f_{for} = f_{\text{инициализации}} + f_{\text{сравнения}} + N(f_{\text{тела}} + f_{\text{инкремента}} + f_{\text{сравнения}})
	\end{equation}
	\item Трудоемкость вызова функции равна 0.
\end{enumerate}

\clearpage
\section{Трудоемкость алгоритмов}
В этой части рассчитаны трудоемкости рассматриваемых алгоритмов.

\subsection{Классический алгоритм}
Трудоемкость классического алгоритма умножения матриц состоит из следующих элементов:
\begin{itemize}
	\item трудоемкость внешнего цикла $i \in [0...n)$:
		\begin{equation}
			f = 2 + n \cdot (2 + f_{\text{ц1}});
		\end{equation}
	\item трудоемкость цикла $j \in [0...s)$:
		\begin{equation}
			f_{\text{ц1}} = 2 + s \cdot (2 + f_{\text{ц2}});
		\end{equation}
	\item трудоемкость цикла $k \in [0...m)$:
	\begin{equation}
		f_{\text{ц2}} = 2 + 13m.
	\end{equation}
\end{itemize}

Суммарная трудоемкость классического алгоритма умножения матриц:
\begin{equation}
	f_{\text{класс}} = 2 + n(4 + s(4 + 13m)) = 13nms + 4ns + 4n + 2 \approx 13nms 
\end{equation}


\subsection{Алгоритм Винограда}
Трудоемкость алгоритма Винограда умножения матриц состоит из следующих элементов:
\begin{itemize}
	\item трудоемкость первого двойного цикла $j \in [0...\frac{m}{2})$ для $i \in [0...n)$:
	\begin{equation}
		f_1 = 2 + n(3 + \frac{m}{2}(3 + 12));
	\end{equation}
	\item трудоемкость второго двойного цикла $j \in [0...\frac{m}{2})$ для $i \in [0...s)$:
	\begin{equation}
		f_2 = 2 + s(3 + \frac{m}{2}(3 + 12));
	\end{equation}
	\item трудоемкость тройного цикла  $k \in [0...\frac{m}{2})$ для $j \in [0...s)$ для $i \in [0...n)$:
	\begin{equation}
		f_3 = 2 + n(2 + s(2 + 7 + 3 + \frac{m}{2}(3 + 23)));
	\end{equation}
	\item трудоемкость оператора выбора с двойным циклом $j \in [0...s)$ для $i \in [0...n)$ (в худшем случае):
	\begin{equation}
		f_{\text{усл}} = 2 + 
		\begin{cases}
			0, & \text{в лучшем случае}\\
			2 + n(2 + s(2 + 13)), & \text{в худшем случае}\\
		\end{cases}.
	\end{equation}
\end{itemize}

Суммарная трудоемкость алгоритма Винограда умножения матриц в лучшем случае:
\begin{equation}
	f_{\text{вин\_лудш}} = 13nms + 12ns + \frac{15}{2}mn + \frac{15}{2}ms + 5n + 3s + 8 \approx 13nms
\end{equation}

Суммарная трудоемкость алгоритма Винограда умножения матриц в худшем случае:
\begin{equation}
	f_{\text{вин\_худш}} = 13nms + 27ns + \frac{15}{2}mn + \frac{15}{2}ms + 5n + 3s + 10 + 2n\approx 13nms
\end{equation}


\subsection{Оптимизированный алгоритм Винограда}
В ходе оценки трудоемкости используются следующие обозначения:
\begin{itemize}
		\item $m_{05} = \frac{m}{2}$
		\item $j_{05} = \frac{j}{2}$
		\item $j_{051} = j_{05} + 1$
\end{itemize}

Трудоемкость оптимизированного алгоритма Винограда умножения матриц состоит из следующих элементов:
\begin{itemize}
	\item трудоемкость первого двойного цикла $j \in [0...m_05)$ для $i \in [0...n)$:
	\begin{equation}
		f_1 = 2 + n(2 + \frac{m}{2}(2 + 10));
	\end{equation}
	\item трудоемкость второго двойного цикла $j \in [0...m_05)$ для $i \in [0...s)$:
	\begin{equation}
		f_2 = 2 + s(3 + \frac{m}{2}(2 + 10));
	\end{equation}
	\item трудоемкость тройного цикла  $k \in [0...m_05)$ для $j \in [0...s)$ для $i \in [0...n)$:
	\begin{equation}
		f_3 = 2 + n(2 + s(2 + 7 + 3 + \frac{m}{2}(2 + 18)));
	\end{equation}
	\item трудоемкость оператора выбора с двойным циклом $j \in [0...s)$ для $i \in [0...n)$ (в худшем случае):
	\begin{equation}
		f_{\text{усл}} = 2 + 
		\begin{cases}
			0, & \text{в лучшем случае}\\
			2 + n(2 + s(2 + 10)), & \text{в худшем случае}\\
		\end{cases}.
	\end{equation}
\end{itemize}

Суммарная трудоемкость оптимизированного алгоритма Винограда умножения матриц в лучшем случае:
\begin{equation}
	f_{\text{опт\_лучш}} = 10nms + 12ns + 6mn + 6ms + 5n + 3s + 8 \approx 10nms
\end{equation}

Суммарная трудоемкость оптимизированного алгоритма Винограда умножения матриц в худшем случае:
\begin{equation}
	f_{\text{опт\_худш}} = 10nms + 12ns + 6mn + 6ms + 5n + 3s + 10 + 2n + 12ns\approx 10nms
\end{equation}

\section*{Вывод}
На основе теоретических данных, полученных в аналитическом разделе были построены схемы исследуемых алгоритмов, посчитаны теоретические трудоемкости алгоритмов в лучших и худших случаях.